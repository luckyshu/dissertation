%
% Modified by Sameer Vijay
% Last Change: Tue Jul 26 2005 13:00 CEST
%
%%%%%%%%%%%%%%%%%%%%%%%%%%%%%%%%%%%%%%%%%%%%%%%%%%%%%%%%%%%%%%%%%%%%%%%%
%
% Sample Notre Dame Thesis/Dissertation
% Using Donald Peterson's ndthesis classfile
%
% Written by Jeff Squyres and Don Peterson
%
% Provided by the Information Technology Committee of
%   the Graduate Student Union
%   http://www.gsu.nd.edu/
%
% Nothing in this document is serious except the format.  :-)
%
% If you have any suggestions, comments, questions, please send e-mail
% to: ndthesis@gsu.nd.edu
%
%%%%%%%%%%%%%%%%%%%%%%%%%%%%%%%%%%%%%%%%%%%%%%%%%%%%%%%%%%%%%%%%%%%%%%%%


%
% Chapter 1
%

\chapter{INTRODUCTION}

\section{Overview}

Over the past few years, a vast array of wireless devices and services have emerged that are fundamentally transforming how we as a society gather and react to information.  Furthermore, the new wireless ecosystem has increased wireless data consumption at phenomenal rates with the most popular cited estimates slating traffic to double every year for the next five years \cite{CiscoAnnualCellGrowth}.  Dubbed the \emph{wireless data tsunami}, the dominant question for wireless service providers (carriers) is how to meet what appears to be an insatiable need for wireless data.  Unlike wired networks, spectrum available for wireless data is finite and typically entails massive costs for acquisition and infrastructure deployment.  Although
technologies such as LTE herald the arrival of fourth-generation (4G) wireless technology, the new speeds often only temporarily satiate the need for additional bandwidth. A wide variety of solutions have emerged ranging simpler solutions such as better WiFi offloading to much more complex solutions such as small heterogeneous cellular networks. For many cellular providers, WiFi offloading, i.e. the users receiving data from 802.11-based hotspots, offers a significant appeal by reducing the strain on the already overloaded cellular infrastructure. Recent studies such as the one in \cite{lee2010mobile} points to offloading offering gains approaching 65\% of the total traffic volume. There are other works such as \cite{balasubramanian2010augmenting, dimatteo2011cellular, han2011mobile, icc2012performance} that discuss the feasibility of WiFi offloading. 

Besides WiFi offloading, a rich category of work that is complementary to existing techniques is the concept of \emph{opportunistic communication}. Opportunistic communication refers to the concept which allows nodes to leverage sporadic, intermittent contacts when two nodes come into direct radio communication range~\cite{pelusi2006opportunistic}. When cellular or WiFi links are not available or not strong enough, opportunistic relaying introduces another alternative option for mobile device to get connected by working in tandem with one or more devices.  From a conceptual standpoint for opportunistic networking, the design of the relaying protocol is critical, namely how does one select and manage appropriate relaying nodes as relays~\cite{laneman2004cooperative,sendonaris2003user,bletsas2006simple,lu2009design,bahl2009opportunistic}.  From a practical standpoint for opportunistic networking, it is essential to prove the potential for opportunistic relaying amongst actual mobile device users in real world. 

The primary goal of my work is to gather high quality smartphone data and leverage the data for reinforcing analysis from a technical network system perspective with respect to network connectivity and performance enhancement. In particular, I answered three interrelated questions: \emph{(i) Can existing wireless technologies on smartphone provide accurate face-to-face proximity estimation and to what extent are relationship formation and maintenance dynamics modified with the introduction of digital communication? (ii) Can the WiFi offloading be the ultimate solution for the predicted wireless data tsunami? (iii) Can the peer-to-peer communication between mobile devices in close proximity be a good candidate for offloading cellular systems?} The answers to these questions are fascinating to explore. The NetSense project is a study of first-year students tracked over a two-year period with respect to nearly all the smartphone information. The process of gathering data results in a rich pool of digital data which provides us the opportunities to analyze the technical dynamics of the network. These involve for instance, the wireless signal strengths can reflect the relative distance and connection status. Moreover, by collecting the information of traffic consumed by the phones, we are able to compare different types of traffic usage and understand their impacts on campus wireless networks. 

\section{Contributions}
The key contributions of my dissertation is as follows:
\begin{itemize}
\item Smartphone monitoring system: By collecting data on Android smartphones, we get a fully anonymized dataset containing device information and digital communication data of participants. It is the foundation of the research and provides multiple dimensions of data for research in both network science and sociology. 

\item Bluetooth proximity: Under the premise that Bluetooth has better accuracy for short distance without the constrain of environment, we use Bluetooth signal strength as an indicator for relative distance between devices. In order to get accurate estimation of face-to-face distance, a proximity estimation model is proposed by leveraging the raw Bluetooth Received Signal Strength Indicator (RSSI) values and introducing multiple thresholds for different environments. 

\item WiFi offloading: For most cellular providers, WiFi offloading appeals to reduce the strain on the overloaded cellular infrastructure. However, based on our data, the WiFi traffic takes approximately 30\% of the total data consumption which is much lower than expected. We analyze the possible reasons for this as well as the usage patterns of users in different categorization according to their WiFi traffic consumption. 

\item Proximity relay: 
With the support of peer-to-peer communication in smartphones, it is practical to leverage the device-to-device data transfers to bridge the coverage gap between cellular and WiFi infrastructure network. Based on our data, we will do the quantitative analysis of the potential relaying which is a key missing part in relaying related research and evaluate the benefits of relaying for cellular traffic offloading in real life. 
\end{itemize}

The rest of this paper is organized as follows. Section~\ref{sec:related_work} provides an overview of the related work and Section~\ref{sec:dataset} introduces the dataset we have. Based on the dataset, the framework to evaluate the potential for relaying is proposed in Section~\ref{sec:potential}. In the end, Section~\ref{sec:conclusion} concludes the paper.
