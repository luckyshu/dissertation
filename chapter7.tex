\chapter{Conclusion and Outlook}
\label{chap:conclusion}

In summary, our work is one of the first longitudinal and fine-grained studies that explores the practical potential for opportunistic communications with respect to smart devices.  Whereas the natural intuition would be cast skepticism towards the potential for opportunistic communications (relaying or collaborative), we demonstrate through our explorations that opportunities for enhancements are not only prevalent but also sufficient, symmetric, and utilizable in our studied environment. Moreover, we find that such opportunities are stable in terms of having sufficient duration, low churn in peers, and relative consistency of peers. Finally, and perhaps most intriguingly, we show that the opportunities exhibit a reasonable degree of reciprocity for the short-term across weekdays and over the long-term when viewed across the entire week.

While we believe our work shows promising capabilities for practical mobile-to-mobile optimizations, the work is merely a first step that deserves further attention from the research community.  Notably, several key aspects need to be explored including: (1) to what extent could one instrument a community outside of a campus drawing on either the Nokia Data Challenge or new efforts to explore a variety of audiences; (2) to what impact might the scale of coverage play a role, for instance would doubling or tripling the population yield considerably better results; And would other techniques besides Bluetooth and WiFi introduce more proximity opportunities (3) are there opportunities to improve content selection and distribution be that through Content Centric Networks (CCN) or other techniques (redundancy elimination); and (4) to what extent is there a `critical mass' where opportunistic networks would seem useful either by analyzing subsets of the data or through theoretical explorations.   
